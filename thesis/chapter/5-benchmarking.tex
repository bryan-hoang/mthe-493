\providecommand{\main}{..}
\documentclass[../mthe-493-final-project.tex]{subfiles}

% Literature review on benchmarking, sourced primarily from Duncan's work

\begin{document}
    \chapter{Benchmarking}
    \label{ch:benchmarking}

    \section{Cycle Time \& Benchmarks}

    TODO: Need to make this section more formal, and include proper citations to Duncan's work

    As specified by Duncan's work, each worker $w_i$ will be benchmarked to estimate

    \begin{enumerate}
        \item an \textit{upload/download rate}, $b_i$
        \item a \textit{sample compute rate}, $C_i$
    \end{enumerate}

    Each worker also has a \textit{fee} to compute a sample, $c_i$. Together, we can define the following quantities:

    \begin{itemize}
        \item \textit{Subjob download time}, $t^d_i$, which is a function of $s_i$
              \[t^d_i = \frac{P_m + P_d s_i}{b_i}\]
        \item \textit{Subjob compute time}, $t^c_i$, which is a function of $s_i$
              \[t^c_i = \frac{\tau s_i}{C_i}\]
        \item \textit{Subjob upload time}, $t^u_i$, which is fixed for a given job
              \[t^u_i = \frac{P_m}{b_i}\]
    \end{itemize}

    A worker's \textit{global cycle time} to compute a subjob is

    \[t_i = t^d_i + t^c_i + t^u_i\]

    The upper bound on training time yields an inequality

    \[t_i < T \quad \forall w_i \in \mathbf{W}\]

    from which an upper bound on \textit{subjob size} is derived

    \[s^{max}_i < \frac{T - \frac{2 P_m}{b_i}}{\frac{\tau}{C_i} - \frac{P_d}{b_i}}\]

    Note that these details are internal to Duncan's research and implementation. We intend to use this implementation to determine $s^{max}_i$.

\end{document}
