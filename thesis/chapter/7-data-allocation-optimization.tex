\providecommand{\main}{..}
\documentclass[../mthe-493-final-project.tex]{subfiles}

%Algorithm implementation. Includes all tested algos and which was selected and why. Visualizes algo's approach to the optimum. Describes code implementation.

\begin{document}
    \chapter{Data Allocation Optimization}
    \label{ch:data-allocation-optimization}
    
    Note from Jared: I moved this here from chapter 2
    
    \section{Qualifying Assumptions}
    The system is designed so that the following assumptions are met to ensure the feasibility of a solution to the optimization problem.
    
    
    \begin{enumerate}
        \item $k \geq \beta$ where $\beta$ is set as the minimum amount of workers to be distributed to.
        \item $n \geq s^{min} \cdot k$
        \item $s_i^{max} \geq s^{min}$ for $i = 1,..,k$
        \item $n \leq \sum_{i=1}^k s_i$, ensuring there is enough combined compute capacity to accept the entire data set.

    \end{enumerate}
    
    \section{Implementation}
    
    As discussed in the midterm presentation, multiple ILP algorithms will be researched, implemented, tested, and evaluated. A preliminary list of algorithms includes
    
    \begin{enumerate}
        \item LP relaxation of the ILP problem (e.g., simplex, dual simplex)
        \item Branch and cut method with LP (e.g., simplex, dual simplex)
        \item Hill climbing technique
    \end{enumerate}
    
    \section{Heuristic Algorithm}
    The following is a heuristic algorithm for achieving an optimal solution to the problem.
    
    \subsection*{Assumptions without loss of generality}
    \begin{enumerate}
        \item $s_i^{max}$, $s_{min}$, $n$, and $k$ are nonnegative integers.
        \item $0 \leq c_1 \leq c_2 \leq ... \leq c_k$ Otherwise rearrange the terms.
    \end{enumerate}
    
    \subsection*{Procedure}
    \begin{enumerate}
        \item If \[ \sum_{i=1}^k s_i = n, \] then $x_i = s_i$, $i = 1,...,k$ is the optimal solution.
        \item If \[\sum_{i=1}^k s_i > n,\] then let $x_i = s^{min}$ for $i = 1,...,k$.
        \item Set $l = 1$. Add 1 to $x_l$ until $x_l = s_l^{max}$, then let $l = l+1$ and repeat. Continue until \[ \sum_{i=1}^{k} x_i = n \]. This yields the optimal solution.
        
    \end{enumerate}

\end{document}