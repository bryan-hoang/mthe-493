\documentclass[../mthe-493-project-proposal.tex]{subfiles}

% Problem Description: By the proposal stage, you should have an application. Based on your intro/background you should be able to abstract it to your application and what your intentions are with your project (and maybe some TBL factors you will be addressing). Some will be very clear, but others may need revising as the actual application is not defined, yet (but try to be as clear as possible).
\begin{document}
    \chapter{Problem Description}
    \label{ch:problem-description}
    We introduce a \textit{network} as a system of heterogeneous units, each with a capacity to produce or utilize some amount of a quantifiable resource on behalf of other units in the network.

    A quantity of system resources take values in $\mathbb{T} \subseteq \mathbb{R}_{\geq 0}$, and are described in relation to a \textit{state space} $\mathbb{X}$ consisting of sequences of finite length, and a set of \textit{work functions} which operate on the state space:
    \begin{equation*}
        \mathbb{W} = \{f: \mathbb{X} \rightarrow \mathbb{X}\}
    \end{equation*}
    To quantify the size of a datum, we introduce a norm on $\mathbb{X}$, denoted $\norm{\cdot}_{\mathbb{X}}$, given by
    \begin{equation*}
        \norm{\{x_1, x_2, ..., x_N\}}_{\mathbb{X}} = N, \forall \{x_1, x_2, ..., x_N\} \in \mathbb{X}
    \end{equation*}
    \textit{Resource production rate}, $R$, is a time-varying stochastic random variable that represents the ability of a unit to produce or utilize a quantity of resources $x \in \mathbb{T}$ per unit time:
    \begin{align*}
        R \colon \mathbb{R} &\to \mathbb{T} \\
        t                   &\mapsto x
    \end{align*}
    When the argument of $R$ is omitted, it is taken to be $R(t)$ for a given $t \in \mathbb{R}$.

    \textit{Resource cost rate}, $C \in \mathbb{R}_{\geq 0}$, is the price that a unit will charge for production per unit of resources. This rate is fixed, i.e. does not vary with time or other network conditions.

    The network structure can be modelled as a directed graph $\mathbf{G} = (\mathbf{V}, \mathbf{A})$.

    Nodes $\mathbf{V} = (v_1, v_2, ..., v_n)$ represent the units in the network. Denote by $\mathbf{R} = (R_1, R_2, ..., R_n)$ and $\mathbf{C} = (c_1, c_2, ..., c_n)$ the associated resource production and cost rates of nodes, respectively.

    The adjacency matrix represents the communication structure of the network, and is given by
    \begin{align*}
        \mathbf{A} \colon \mathbb{R} &\to \mathbb{R}^{n \times n}_{\geq 0} \\
        t                            &\mapsto A
    \end{align*}
    An edge $\mathbf{A}_{i,j}$ is a time-varying stochastic random variable with finite mean representing the \textit{communication cost} from $v_i$ to $v_j$. The communication cost is proportional to an amount of data transferred between nodes. When the argument of $\mathbf{A}$ is omitted, it is taken to be $\mathbf{A}(t)$ for a given $t \in \mathbb{R}$.

    Some assumptions made about $\mathbf{A}$ in this model: for all $i, j \in \{1, 2, ..., n\}$,

    \begin{enumerate}
        \item If $\mathbf{A}_{i,j} = 0$, there is no communication from $v_i$ to $v_j$
        \item If $\mathbf{A}_{i,j} > 0$, then $\mathbf{A}_{j,i} > 0$ (communication is always bidirectional)
        \item We take $\mathbf{A}_{i,i} = 0$ (no self-communication)
        \item The network communication structure is stable; that is, if there exists $T \in \mathbb{R}$ such that $\mathbf{A}_{i,j}(T) > 0$, then $\mathbf{A}_{i,j}(t) > 0$ for all $t \in \mathbb{R}$
    \end{enumerate}

    A \textit{job} is the pairing of a work function and a finite tuple of \textit{input data} from which job \textit{results} can be derived. Formally, a job is
    \begin{equation*}
        J = (f, D) \text{ where } f \in \mathbb{W} \text{ and } D = (d_1, d_2, ..., d_N) \subset \mathbb{X}
    \end{equation*}
    and the tuple of results is
    \begin{equation*}
        Y = (f(d) \mid d \in D) \subset \mathbb{X}.
    \end{equation*}
    A \textit{job deployer} is a node that employs some or all of the adjacent nodes to perform a job. It is assumed that all units on the network are capable of deploying and executing a job.

    Job deployment proceeds as follows: denote by $V_i$ the set of nodes adjacent to $v_i$, given by
    \begin{equation*}
        V_i = \{V_j \in V \mid \mathbf{A}_{i,j} > 0\}
    \end{equation*}
    The job is deployed to a subset of \textit{workers}, $E \subseteq V_i$, of adjacent nodes by partitioning the input data into disjoint subsets such that
    \begin{equation*}
        D = \bigcup\limits_{j \mid v_j \in E} D_j
    \end{equation*}
    This partitions $J$ into a set of \textit{subjobs} $\overline{J} = \{J_j = (f, D_j) \mid v_j \in E\}$, such that $v_j$ is assigned subjob $J_j$. The results of the subjob $Y_j = (f(d) \mid d \in D_j)$ are returned to $v_i$ upon completion. Note that subjobs are also jobs, and are assumed to be executable independently of one another

    Some performance oriented cost functions are defined as follows: given network $\mathbf{G} = (\mathbf{V}, \mathbf{A})$, and job $J = (f, D)$

    \begin{itemize}
        \item The \textit{execution cost} of the work function $f$ on a datum $d \in D$ to      produce $f(d) \in Y$ is a function $\xi_E$
              \begin{align*}
                  \xi_E \colon \mathbb{W} \times \mathbb{X} &\to \mathbb{T} \\
                  (f, d)                                    &\mapsto c
              \end{align*}

        \item \textit{Total execution cost} of job $J$ is the function $\xi_{TE}$
              \begin{align*}
                  \xi_{TE} \colon \mathbb{W} \times D \subset \mathbb{X} &\to \mathbb{T}                            \\
                  (f, D)                                                 &\mapsto \sum\limits_{d \in D} \xi_E(f, d)
              \end{align*}

        \item \textit{Network deploy cost} is the network cost of deploying job $J$ from $v_i$ to $v_j$ given $\mathbf{A}$ is
              \begin{equation*}
                  \gamma_{ND}(J, j, i) = \mathbf{A}_{i,j} \sum\limits_{d \in D_j} \norm{d}_{\mathbb{X}}
              \end{equation*}
              The above equation assumes that the network cost of transmitting the work function is negligible relative to the size of the input data. Because $\mathbf{A}_{i,j}$ is a random variable, $\gamma_{ND}$ is stochastic in nature.

        \item \textit{Network result cost} is the network cost of receiving results of job $J$ from     $v_j$ to $v_i$ given $\mathbf{A}$ is
              \begin{equation*}
                  \gamma_{NR}(J, j, i) = \mathbf{A}_{j,i} \sum\limits_{d \in D} \norm{f(d)}_{\mathbb{X}}
              \end{equation*}
              Because $\mathbf{A}_{j,i}$ is a random variable, $\gamma_{NR}$ is stochastic in nature.

        \item Given the preceding two definitions, the \textit{network cost} of deploying job $J$ from $v_i$ to $v_j$ and receiving the results  given $\mathbf{A}$ is
              \begin{equation*}
                  \gamma_N(J, j, i) = \gamma_{ND}(J, j, i) + \gamma_{NR}(J, j, i)
              \end{equation*}
              Because $\gamma_{ND}$ and $\gamma_{NR}$ are stochastic in nature, so it $\gamma_N$.
    \end{itemize}

    When we are additionally given production rates $\mathbf{R}$, cost rates $\mathbf{C}$, a job deployer $v_i \in \mathbf{V}$, workers $E \subseteq V_i$, and subjobs $\overline{J} = \{J_j \mid v_j \in E\}$, we can define the following cost functions:

    \begin{itemize}
        \item \textit{Total network cost} when given $\mathbf{A}$ is
              \begin{align*}
                  \gamma_T(\overline{J}, E, i) &= \sum_{j \mid v_j \in E} \gamma_N(J_j, j, i) \\
                                               &= \sum_{j \mid v_j \in E} \left(
                  \mathbf{A}_{i,j} \sum\limits_{d \in D_j} \norm{d}_{\mathbb{X}} + \mathbf{A}_{j,i} \sum\limits_{y \in Y_j} \norm{y}_{\mathbb{X}}
                  \right)
              \end{align*}
              Note that $\gamma_T$ is stochastic in nature.

        \item The \textit{completion time}, denoted $t_C$, is the maximal sum of expected subjob execution time and expected network costs. Formally,
              \begin{align*}
                  t_C &= \max_{j \mid v_j \in E}
                  \left(
                  \mathbb{E}\left[ \frac{\xi_{TE}(J_j, j, i)}{R_j} \right]
                  + \mathbb{E}\left[ \gamma_N(J_j, j, i) \right]
                  \right)                        \\
                      &= \max_{j \mid v_j \in E}
                  \left(
                  \mathbb{E}\left[ \sum\limits_{d \in D_j} \frac{\xi(f, d)}{R_j}\right]
                  + \mathbb{E}\left[
                      \mathbf{A}_{i,j} \sum\limits_{d \in D_j} \norm{d}_{\mathbb{X}} + \mathbf{A}_{j,i} \sum\limits_{y \in Y_j} \norm{y}_{\mathbb{X}}
                      \right]
                  \right)
              \end{align*}
        \item \textit{Resource cost}, denoted $c_R$, is the sum of costs arising from the use of worker resource production, based on total execution costs of subjobs and resource cost rates. $c_R$ is given by
              \begin{equation*}
                  c_R(\overline{J}, E, \mathbf{C}) = \sum_{j \mid v_j \in E} C_j \cdot \xi_{TE}(J_j)
              \end{equation*}
    \end{itemize}

    For a given network $\mathbf{G} = (\mathbf{V}, \mathbf{A})$ and job $J$, the particular selection of workers $E$ and subjob partition $\overline{J}$ allows for a mechanism of control over completion time $t_C$ and resource cost $c_R$. The uncertainty present in adjacency matrix $\mathbf{A}$ and production rates $\mathbf{R}$ allow us to formulate the manipulation of $t_C$ and $c_R$ as a stochastic optimization problem. It is of particular interest to optimize for these quantities under various circumstances, such as

    \begin{itemize}
        \item Minimization of completion time $t_C$ with unconstrained resource cost $c_R$
        \item Minimization of resource cost $c_R$ with unconstrained completion time $t_C$ (although this is trivial, since values in question are deterministic)
        \item Minimization of completion time $t_C$ for a fixed resource cost $c_R$
        \item Minimization of resource cost $c_R$ for a fixed completion time $t_C$
    \end{itemize}

    The last set of circumstances will be the focus of the subsequent application.

    \section{Application}
    % In the application section, apply the formal math from above in the specific context of our distributed computing problem (based on intro/background section etc.)

    \textit{Distributed computing} on a computer network can be modelled using the notion of a network described above. A computer network can be configured in either a \textit{decentralized} or \textit{centralized} way. In a decentralized network, edges can be created between any two nodes. There is no controlling node or any central authority which dictates the flow of information through the entire network.

    In centralized computing, a network is configured with a hierarchical edge structure and the network is controlled at a central authoritative node which directly connects to all other nodes. In distributed computing, this central authority is called an \textit{orchestrator}. The orchestrator has a direct connection with each node on the network and when acting as a \textit{job deployer} it can direct and control distributed computing tasks.

    \section{Triple Bottom Line Factors}
    \blindtext
\end{document}
